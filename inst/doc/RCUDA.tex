\documentclass[article]{jss}

%%%%%%%%%%%%%%%%%%%%%%%%%%%%%%
%% declarations for jss.cls %%%%%%%%%%%%%%%%%%%%%%%%%%%%%%%%%%%%%%%%%%
%%%%%%%%%%%%%%%%%%%%%%%%%%%%%%

%% almost as usual
\author{Paul Baines\\University of California at Davis\\
        Duncan Temple Lang\\University of California at Davis}
\title{RCUDA: General programming facilities for GPUs in R}

%% for pretty printing and a nice hypersummary also set:
\Plainauthor{Paul Baines, Duncan Temple Lang}
\Plaintitle{RCUDA: General programming facilities for GPUs in R}
\Shorttitle{RCUDA}

%% an abstract and keywords
\Abstract{ General Purpose Graphical Programming Units (GPGPUs)
  provide the ability to perform computations in a massively parallel
  manner.  Their potential to significantly speed computations for
  certain classes of problems has been clearly demonstrated.  We
  describe an \proglang{R} package that provides high-level
  functionality that allows \proglang{R} programmers to experiment
  with and exploit NVIDIA GPUs.  The package provides an interface to
  the routines and data structures in the CUDA software development
  kit (SDK), and also provides higher-level \proglang{R} interfaces to
  these. Currently, programmers write the code that runs on the GPU in
  \proglang{C} code but call that code and manage the inputs and
  outputs in \proglang{R} code. We describe experimental approaches to
  compile \proglang{R} directly so that all computations can be
  expressed in \proglang{R}.
}
\Keywords{GPUs, parallel computing, \R, \Rpkg{RCUDA} package}
\Plainkeywords{GPUs, parallel computing, R, RCUDA} %% without formatting
%% at least one keyword must be supplied

%% publication information
%% NOTE: Typically, this can be left commented and will be filled out by the technical editor
%% \Volume{13}
%% \Issue{9}
%% \Month{September}
%% \Year{2004}
%% \Submitdate{2004-09-29}
%% \Acceptdate{2004-09-29}

%% The address of (at least) one author should be given
%% in the following format:
\Address{
  Paul Baines\\
  4210 Math Sciences Building, \\
  University of California at Davis \\
  One Shields Avenue\\
  Davis, CA 95616\\
  E-mail: \email{pbaines@ucdavis.edu}\\
  URL: \url{http://www.stat.ucdavis.edu/~pdbaines/}
\\
\\
  Duncan Temple Lang\\
  4210 Math Sciences Building, \\
  University of California at Davis \\
  One Shields Avenue\\
  Davis, CA 95616\\
  E-mail: \email{duncan@r-project.org}\\
  URL: \url{http://www.omegahat.org}
}

\def\C{\proglang{C}}
\def\R{\proglang{R}}
\def\llvm{\proglang{LLVM}}
\def\Rpkg#1{\pkg{#1}}
\def\Rfunc#1{\textsl{#1()}}
\def\Rvar#1{\textsl{#1}}
\def\Cfunc#1{\textit{#1()}}
\def\Cvar#1{\textit{#1}}
\def\file#1{\textbf{#1}}
\def\Ctype#1{\texttt{#1}}
\def\Rclass#1{\textit{#1}}
\def\Rarg#1{\textbf{#1}}

\def\ShFlag#1{\textit{#1}}

\DefineVerbatimEnvironment{RCode}{Verbatim}{fontshape=sl}
\DefineVerbatimEnvironment{CCode}{Verbatim}{fontshape=tt}
\DefineVerbatimEnvironment{ShCode}{Verbatim}{fontshape=it}

\begin{document}

\section{Introduction}\label{sec:Introduction}
GPUs have emerged as a computational platform for massively parallel
computations. Unlike CPUs, their performance continues to grow
dramatically.  They are very good for particular classes of
computations, some quite common in statistics.  It is important for
statisticians to be able to leverage these so that we have another
approach to make computations faster and so scale to larger data sets
or simulations.

GPUs require the programmer to use a different computational model
when implementing an algorithm.  They also use a slightly extended
version of the  \C{}  programming language and require
the programmer to think about two processing units - the host CPU
and the  GPU  device. The programmer must allocate and move data from the
host to the device and back.  This consumes resources.


The \pkg{RCUDA} package is quite different in intent and functionality
than other GPU-related \R packages such as
\Rpkg{gputools}~\cite{bib:gputools} and \Rpkg{rgpu}~\cite{bib:rgpu}.
\Rpkg{gputools} implements several commonly used statistical
algorithms in \C{} code that runs on a GPU and provides \R{} functions
to invoke those with data in \R.  The set of functions is fixed and
\R{} programmers wanting to implement a different algorithm or
approach for one of these algorithms must program in \C.

The \Rpkg{rgpu} package also provides implementation of a few
algorithms written in \C{} that run on a GPU.  However, it also
provides an ``interpreter'' for \R{} scalar mathematical expressions.
This does not appear to handle arbitrary code % in spite of its claims
and also has to map each expression from \R{} to a different
representation for each call from \R{} to the GPU.
It also interprets this on the GPU rather than using native code. 
% It  also doesn't install cleanly.  
We discuss in section~\ref{sec:LLVM} how might be able to compile a
larger subset of the \R{} language to native GPU code.

The \Rpkg{RCUDA} package is similar in nature to
\Rpkg{OpenCL}~\cite{bib:ROpenCl}.  It targets the CUDA SDK and NVIDIA
GPUs. This is based on the belief that the dedicated SDK can
outperform the more general OpenCL~\cite{bib:OpenCL} SDK.  Also,
\Rpkg{RCUDA} aims to expose the entire SDK to \R{} programmers. The
\Rpkg{OpenCL} package provides the essential functionality to invoke
kernels on the GPU, passing data from \R{} to the GPU and back.  This
difference illustrates one of the primary motivations of the
\Rpkg{RCUDA} package. We want \R{} programmers to experiment with
different features of the SDK and to explore the performance
characteristics of various programming strategies for GPUs.  Different
GPUs exhibit quite different performance characteristics, and
different programming paradigms and even tuning parameters also can
have significant impact. For this reason, we want to be able to be
able to control these aspects dynamically in a high-level language 
rather than fix them statically in a language such as \C.




\section[The Basics of the RCUDA Package]{The Basics of the \Rpkg{RCUDA} package}\label{sec:}
In this section, we describe both the essential concepts of the CUDA
SDK and the \R{} interface to it provided by the \Rpkg{RCUDA} package.

At its very simplest, computing with GPUs involves 
\begin{enumerate}
\item writing (and compiling) a kernel that performs the computations,
\item allocating memory on the GPU device, 
\item copying data from the host to the device
\item invoking the kernel
\item copying the results from the device to the host.
\end{enumerate}
We will discuss each of these steps.

For now, we require that the kernel be compiled outside of \R.  This
means that an \R{} programmer can write and compile a kernel or
perhaps that it be compiled by somebody else and made available to the
\R{} programmer.  The \Rpkg{RCUDA} package can compile a kernel for
the \R{} user but it is often more convenient to use the command-line
to compile the code directly. 
We'll use the \Cfunc{dnorm} kernel as an example.
Our kernel code is reasonably simple for a CUDA kernel and is shown in
figure~\ref{fig:dnormKernelCode}.
 % should we show the code for the kernel?
% This probably requires a separate section  describing the basic idea
% of kernels and how they  know what to operate on and where to put
% their results. threadIdx, blockIdx, blockDim, gridDim. This requires
% some explanation,
\begin{figure}
\begin{CCode}
extern "C"
__global__ void dnorm_kernel(float *vals, int N, float mu, float sigma)
{
       // Taken from geco.mines.edu/workshop/aug2010/slides/fri/cuda1.pd
    int myblock = blockIdx.x + blockIdx.y * gridDim.x;
           /* how big is each block within a grid */
    int blocksize = blockDim.x * blockDim.y * blockDim.z;
            /* get thread within a block */
    int subthread = threadIdx.z*(blockDim.x * blockDim.y) + 
                         threadIdx.y*blockDim.x + threadIdx.x;

    int idx = myblock * blocksize + subthread;

	if(idx < N) {
            float std = (vals[idx] - mu)/sigma;
	    float e = exp( - 0.5 * std * std);
	    vals[idx] = e / ( sigma * sqrt(2 * 3.141592653589793));
	}
}
\end{CCode}
\caption{The \C{} code defining a GPU kernel to compute the Normal
  density.  This kernel writes the results back into the input
  vector, overwriting its values.
}\label{fig:dnormKernelCode}
\end{figure}

Before we can invoke a kernel, we have to load it into the host
process, i.e. \R.  We do this by loading the compiled code as a
\textit{Module}.  We use the \R{} function \Rfunc{loadModule} to do
this. It can read the code in various formats - the PTX text format
and the two binary formats cubin and fatbin.

We compile it with 
\begin{ShCode}
nvcc --ptx -o dnorm.ptx dnorm.cu
\end{ShCode}
Alternatively, we can compile the two binary formats with 
\begin{ShCode}
nvcc -cubin -m64 -gencode arch=compute_10,code=sm_10 -o dnorm.cubin dnorm.cu
nvcc -fatbin -m64 -gencode arch=compute_10,code=sm_10 -o dnorm.fatbin dnorm.cu
\end{ShCode}
respectively.
The \ShFlag{arch} flag specifies the GPU compatability level.%! Fix this

Once we have compiled the code, we can load it with
\begin{RCode}
mod = loadModule("dnorm.ptx")
\end{RCode}
Next, we can obtain a reference to the particular routine in the
module  that we want to invoke.  The \R{} interface makes the module
appear like a list and so we can use
\begin{RCode}
kernel = mod$dnorm_kernel
\end{RCode}
%$ - fool latex mode
to get this reference.

We now have the kernel, but need the data to pass to it.  We'll use a simple example
and simulate it in \R{} via the \Rfunc{rnorm} function.
\begin{RCode}
N = 1e6
mean = 2.3
sd =  2.1
x = rnorm(N, mean, sd)
\end{RCode}

We have outlined the explicit steps involved in calling a kernel.
However, we can shortcut these steps using the \R{} interface.  The
\Rfunc{.cuda} (or the synonomous \Rfunc{.gpu}) will copy an \R{} vector
argument to the device. So we can invoke the kernel with
\begin{RCode}
ans = .cuda(kernel, x, N, mean, sd, gridDim = c(62, 32), blockDim = 512)
\end{RCode}
We must explicitly specify the dimension for the grid and block to be
used when invoking the kernel. Each of these can specify
three-dimensions, however, any omitted dimension defaults to $1$.
We have chose $62$, $32$ and $512$ for this example as the product
of these exceeds the number of elements \Rvar{N}. We will have some
redundant threads on the GPU, but that is cheap.

The \Rfunc{.cuda} function recognizes the inputs and determines which
to copy to the device and which can be passed directly (i.e. the
scalar values \Rvar{N}, \Rvar{mean} and \Rvar{sd}). We'll see later
that we can pass arguments that refer to data or memory already on the
GPU and \Rfunc{.cuda} recognizes that it doesn't need to transfer
this, but merely pass the reference as-is.

The \Rfunc{.cuda} function also recognizes that since the data was
passed from \R{}, it may be modified and returns that vector.  If
there are multiple vector inputs, it returns all of these as list in
the same way the \Rfunc{.C} function behaves. However, \Rfunc{.cuda}
recognizes if a single input argument is a vector and so \Rvar{ans} in
our example contains the actual vector of normal density values.  More
generally, we can specify which inputs are to be copied back to \R{}
from the device via the \Rarg{outputs} parameter of the \Rfunc{.cuda}
function.

\section{Examples/case-Studies and performance}\label{sec:}
% 
% Distance computations using kernel from gputools
% Particle filter
In this section, we look at several examples and see how they perform
on various GPUs.  The first example is a simple scalar version of the
normal density function and we compare this with the highly efficient
\Rfunc{dnorm} function in base \R.  The second example adapts a
function in the \Rpkg{gputools} package to compute the distances
between observations in two data sets.  This illustrates how we can
use an existing kernel and invoke it from \R{} without the supporting
\C{} code to interface from \R.  A third example implements a particle
filter via GPU computations.



\section{Low-level interface and its implementation}
We initially implemented the bindings to copy data between the host
(CPU) and device (GPU) and the ability to launch a kernel to verify
that this approach would work and to deal with the marshalling of \R{}
objects to the kernel calls and vice-versa.  We also added some
additional bindings to query the available GPUs and their properties.
Subsequently, however, we programmatically generated the binding (\R{}
and \C) code and the enumerated constants and information about the
data structures (e.g. the size of each so we can use that when
allocating memory on the device).  

% Show an example of a binding and how it appears.
Next, we describe a reasonably simple example of a programmatically
generated binding to illustrate the nature of the interface and to
help mapping  from the CUDA documentation to the \R{} functions.
One function is \Rfunc{cuDeviceGetAttribute} and corresponds to the routine
with the same name in the CUDA API.
The routine is defined  in the header file \file{cuda.h} as
\begin{Code}
CUresult CUDAAPI 
cuDeviceGetAttribute(int *pi, CUdevice_attribute attrib, CUdevice dev);
\end{Code}
The second parameter identifies which attribute we are querying and
the third parameter identifies the device. The latter is an integer,
with $0$ corresponding to the first device, $1$ for the second and so
on.  The first parameter is a pointer to a single integer. This is how
the CUDA API returns the result of the query.  This is an output
variable. Accordingly, we do not need to include in the \R{} function
that calls the \C{} routine.

The CUDA routine returns a status value of type \Ctype{CUresult}.  We
check this to see if the call was successful. If it was, we return the
result stored in \Cvar{pi}; otherwise, we raise an error in \R, using
the particular value of the status type and the CUDA error message.

The \R{} function we generate to interface to the CUDA routine is
defined as
\begin{RCode}
cuDeviceGetAttribute <-
function( attrib , dev  )
{
   ans = .Call('R_auto_cuDeviceGetAttribute', 
                   as(attrib, 'CUdevice_attribute'), 
                   as(dev, 'CUdevice'))
   if(is(ans, 'CUresult') && ans != 0)
       raiseError(ans, 'R_auto_cuDeviceGetAttribute')
   ans
}
\end{RCode}
This makes the call to the wrapper routine
\Cfunc{R\_auto\_cuDeviceGetAttribute}, passing it the two \R{} values
identifying the attribute and device of interest.  An important step
here is that this \R{} function coerces the arguments it receives to
the correct types.  Here we have an opportunity to use $1$-based
counting familiar to \R{} users for specifying the device.  The
coercion from the device number to a \Rclass{CUdevice} object in \R{}
performs the subtraction to map to the $0$-based couting used by CUDA.
We manually defined the \Rclass{CUdevice} class and the coercion method.

The function checks the status of the result to see if it is an \R{}
object of class \Rclass{CUresult}. If it is and not $0$, the call
produced an error and we raise this in \R. By doing this in \R{}
rather than in \C{} code, it is easier to raise exceptions/conditions
with classes corresponding to the type of error. This allows
programmers to react to particular types of errors, e.g. out of memory
or invalid device, in different ways.

The final piece of the bindings is the \C{} wrapper routine
\Cfunc{R\_auto\_cuDeviceGetAttribute}. This is defined as 
\begin{CCode}
SEXP
R_auto_cuDeviceGetAttribute(SEXP r_attrib, SEXP r_dev)
{
    SEXP r_ans = R_NilValue;
    int pi;
    CUdevice_attribute attrib = (CUdevice_attribute) INTEGER(r_attrib)[0];
    CUdevice dev = INTEGER(r_dev)[0];
    CUresult ans;
    ans = cuDeviceGetAttribute(& pi,  attrib,  dev);
    if(ans)
       return(R_cudaErrorInfo(ans));
    r_ans = ScalarInteger(pi) ;
    return(r_ans);
}
\end{CCode}
The logic is quite straightforward and very similar to how we would
write this manually. We convert the \R-level arguments to their
equivalent \C{} types.  We create a local variable (\Cvar{pi}) used to
retrieve the actual result.  We call the CUDA routine and check the
status value.  If there is an error, we call a manually written
routine \Cfunc{R\_cudaErrorInfo} which collects information about the
CUDA error in the form of an \R{} object.  If the CUDA call was
successful, we return the value of the local variable \Cvar{pi}
converted to the approriate \R{} type.



We used \Rpkg{RCIndex}~\cite{bib:RCIndex} to both read the
declarations in the CUDA header files and to generate the bindings.
Programmatically generating the bindings reduces the number of simple
programming errors and also allows us to update the bindings when new
versions of the API and SDK are released.

The CUDA API uses an idiom for most of its routines.  A routine
returns an error status value, and if this indicates success, the
actual results are accessible via the pointers we passed to the
routine. In other words, the routines return their values via input
pointers.  We specialized the general binding generation mechanism in
\Rpkg{RCIndex} to exploit this idiom and be able to understand what
was essentially an output parameter passed as a pointer input and what
were actual pointer input parameters.  This code to generate the
CUDA-specific bindings is included in the package.




\section[Compiling R code to native GPU code]{Compiling \R{} code to native GPU code}\label{sec:LLVM}
% Move into Future work if we don't get this running.
The following is not yet fully implemented, but is something 
we know to be possible and related to work we are pursuing in a
slightly related project.
\llvm{} stands for the Low Level Virtual Machine
and is a compiler toolkit which we can embed in \R. 
This allows us to dynamically generate machine code within an \R{}
session and invoke those newly created routines, passing them 
inputs from \R{} and obtaining the results in \R.
This is very powerful and allows us to ``compile around the \R{}
interpreter'' and gain significant speedups for some common \R{}
idioms~\cite{bib:StatSciLLVM}.

\llvm{} can generate machine code for the machine's CPU, but can also
create PTX code to run on an NVIDIA GPU.  We can use our \R{} compiler
code (\Rpkg{RLLVMCompile}), or a specialized version of it for
targetting GPUs, and have it generate PTX code for kernel routines.
We can then load these routines onto the GPU and invoke them as a
regular kernels.  This was done using \R{} as an example by
researchers in NVIDIA~\cite{bib:libNVVM} investigating dynamic, interpreted
languages and GPUs.

\section{Future Work}

Now that we have the general-purpose bindings and access to the CUDA
API, we, and hopefully others, will explore how to map statistical
algorithms onto the GPU to take advantage of their potential.  We plan
to explore different examples and approaches and understand their
characteristics.  

We also plan to explore the potential benefits of aspects of the API
such as pinned memory, streams for interleaving computations,
peer-to-peer communication between two or more GPUs.

As the need arises, we may generate bindings to other GPU-related
libraries and APIs. These include There is no need to have \R{}
bindings to interface with kernel-level libraries such as
Thrust~\cite{bib:Thrust} and parts of curand~\cite{bib:curand}.
However, we may need to determine the sizes of the different data
structures they use that should be allocated by the host on the
device.


\bibliography{gpu}


\end{document}
